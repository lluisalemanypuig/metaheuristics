\newcounter{decisionVariables}

\section{Integer Linear Programming formulation}

In this section is described all the information regarding the ILP formulation of the problem: the
variables, objective function, the constraints to model the problem, and the implementation in OPL.

\subsection{Decision variables}

For the ILP formulation of this problem the following decision variables have been defined. The name
each of these decision variables are given in the OPL code are also specified.

\begin{enumerate}
    \item $I_\icentre$: ``centre $\ecentre_\icentre$ is installed" -
    ($\mathbb{B}$, $\forall \icentre: 1 \le \icentre \le \ncentre$)
    \label{dec-var:centre-installed}
    \stepcounter{decisionVariables}
        
    \begin{lstlisting}
    dvar boolean installed_centre[t in T];
    \end{lstlisting}
    
    \item $I_{\icentre, \iloc}$: ``centre $\ecentre_\icentre$ is installed in location
    $\eloc_\iloc$" - ($\mathbb{B}$, $\forall \icentre, \iloc: 1 \le \icentre \le \ncentre,\:
    1 \le \iloc \le \nloc$)
    \label{dec-var:centre-location}
    \stepcounter{decisionVariables}
    
    \begin{lstlisting}
    dvar boolean location_centre[t in T][l in L];
    \end{lstlisting}

    \item $P_{\icentre, \icity}$: ``centre $\ecentre_\icentre$ serves city $\ecity_\icity$ as
    primary centre" - ($\mathbb{B}$, $\forall \icentre, \icity: 1 \le \icentre \le \ncentre,\:
    1 \le \icity \le \ncity$)
    \label{dec-var:primary-centre-city}
    \stepcounter{decisionVariables}
    
    \begin{lstlisting}
    dvar boolean primary_centre[t in T][c in C];
    \end{lstlisting}

    \item $S_{\icentre, \icity}$: ``centre $\ecentre_\icentre$ serves city $\ecity_\icity$ as
    secondary centre" - ($\mathbb{B}$, $\forall \icentre, \icity: 1 \le \icentre \le \ncentre,\:
    1 \le \icity \le \ncity$)
    \label{dec-var:secondary-centre-city}
    \stepcounter{decisionVariables}
    
    \begin{lstlisting}
    dvar boolean secondary_centre[t in T][c in C];
    \end{lstlisting}

\end{enumerate}

For the sake of brevity, we will refer to:

\begin{itemize}
    
    \item $P_\icentre = \{ \ecity_\icity \,|\, P_{\icentre, \icity} = 1, \, 1 \le \icity \le \ncity \}$
    as the set of cities the centre $\ecentre_\icentre \in \centre$ serves as a primary centre.
    
    \item $S_\icentre = \{ \ecity_\icity \,|\, S_{\icentre, \icity} = 1, \, 1 \le \icity \le \ncity \}$
    as the set of cities the centre $\ecentre_\icentre \in \centre$ serves as a secondary centre.
    
    \item $u_{\iloc, x}, u_{\iloc, y}$ as the $x$ and $y$ coordinates of location $\eloc_\iloc \in \loc$.

    \item $v_{\icity, x}, v_{\icity, y}$ as the $x$ and $y$ coordinates of city $\ecity_\icity \in \city$.
    
\end{itemize}

\subsection{Input data}

The input data that makes up an instance of the problem, extracted from the statement, is referenced in
the OPL code using the following names:

\begin{itemize}
    \item[-] Location coordinates $u_\iloc$
    
    \begin{lstlisting}
    float loc_x[l in L];
    float loc_y[l in L];
    \end{lstlisting}
    
    \item[-] City coordinates $v_\icity$:
    
    \begin{lstlisting}
    float city_x[c in C];
    float city_y[c in C];
    \end{lstlisting}
    
    \item[-] City population $p_\icity$:
    
    \begin{lstlisting}
    int city_pop[c in C];
    \end{lstlisting}
    
    \item[-] Centre working distance $\omega_\icentre$:
    
    \begin{lstlisting}
    float work_dist[t in T];
    \end{lstlisting}
    
    \item[-] Centre capacity $s_\icentre$:
    
    \begin{lstlisting}
    int centre_cap[t in T];
    \end{lstlisting}
    
    \item[-] Centre installation cost $i_\icentre$:
    
    \begin{lstlisting}
    float instal_cost[t in T];
    \end{lstlisting}
    
\end{itemize}

\subsection{Objective function}

The value to minimise is the cost of centre installation. Therefore, we will be using the
following objective function:

\begin{equation}
\text{minimise  } \sum_{\icentre = 1}^{\ncentre} I_\icentre \cdot i_\icentre
\label{ILP:objective-function}
\end{equation}

that involves the installed centres $I_\icentre$ and the corresponding installation cost $i_\icentre$.

\hfill

The implementation in OPL is the following:

\begin{lstlisting}
minimize sum(t in T) installed_centre[t]*instal_cost[t];
\end{lstlisting}

\subsection{Constraints}

The problem will be ultimately solved using the following constraints.

\begin{enumerate}
\setcounter{enumi}{\value{equation}}

    \item A location can have at most one centre installed.
    \begin{equation}
    \sum_{\icentre = 1}^{\ncentre} I_{\icentre, \iloc} \le 1
    \qquad \forall \iloc: 1 \le \iloc \le \nloc
    \label{cstr:one-loc-one-centre-1}
    \end{equation}
    
    \begin{lstlisting}
    forall (l in L) {
    	sum (t in T) location_centre[t][l] <= 1;
    }
    \end{lstlisting}

    \item A centre can be installed in at most one location.
    \begin{equation}
    \sum_{\iloc = 1}^{\nloc} I_{\icentre, \iloc} \le 1
    \qquad \forall \icentre: 1 \le \icentre \le \ncentre
    \label{cstr:one-loc-one-centre-2}
    \end{equation}
    
    \begin{lstlisting}
    forall (t in T) {
    	sum (l in L) location_centre[t][l] <= 1;
    }
    \end{lstlisting}

    \item A centre is installed if it is placed at a location.
    \begin{equation}
    I_\icentre = \sum_{\iloc = 1}^{\nloc} I_{\icentre, \iloc}
    \qquad \forall \icentre: 1 \le \icentre \le \ncentre
    \label{cstr:is-centre-installed}
    \end{equation}
    
    \begin{lstlisting}
    forall (t in T) {
    	sum (l in L) location_centre[t][l] == installed_centre[t];
    }
    \end{lstlisting}

    \item All installed centres must serve at least one city, either as primary or as secondary
    centre:
    \begin{equation}
    \sum_{\icity = 1}^{\ncity} (P_{\icentre, \icity} + S_{\icentre, \icity}) \ge I_{\icentre}
    \qquad \forall \icentre: 1 \le \icentre \le \ncentre
    \label{cstr:inst-centres-must-serve}
    \end{equation}
    
    \begin{lstlisting}
    forall (t in T) {
    	sum (c in C)
    	(
    	    primary_centre[t][c]
    	    +
    	    secondary_centre[t][c]
    	)
    	>= installed_centre[t];
    }
    \end{lstlisting}

    \item A centre that is not installed can not serve any city, neither as primary centre or
    secondary.
    \begin{equation}
    P_{\icentre, \icity} \le I_\icentre,
    \quad
    S_{\icentre, \icity} \le I_\icentre
    \qquad \forall \icity, \icentre:
    1 \le \icity \le \ncity,\;
    1 \le \icentre \le \ncentre
    \label{cstr:not-installed-not-prim-sec}
    \end{equation}
    
    \begin{lstlisting}
    forall (t in T, c in C) {
    	primary_centre[t][c] <= installed_centre[t];
    	secondary_centre[t][c] <= installed_centre[t];
    }
    \end{lstlisting}

    \item Each city has to be served exactly by one primary centre and a secondary centre.
    \begin{equation}
    \sum_{\icentre = 1}^{\ncentre} P_{\icentre, \icity} = 1,
    \quad
    \sum_{\icentre = 1}^{\ncentre} S_{\icentre, \icity} = 1
    \qquad \forall c: 1 \le \icity \le \ncity
    \label{cstr:all-cities-served}
    \end{equation}
    
    \begin{lstlisting}
    forall (c in C) {
    	sum (t in T) primary_centre[t][c] == 1;
    	sum (t in T) secondary_centre[t][c] == 1;
    }
    \end{lstlisting}

    \item A centre can not serve a city as both primary and secondary centre to the same city.
    \begin{equation}
    P_{\icentre, \icity} + S_{\icentre, \icity} \le 1
    \qquad \forall \icity, \icentre:
    1 \le \icity \le \ncity,\;
    1 \le \icentre \le \ncentre
    \label{cstr:centre-not-too-busy}
    \end{equation}
    
    \begin{lstlisting}
    forall (t in T, c in C) {
    	primary_centre[t][c] + secondary_centre[t][c] <= 1;
    }
    \end{lstlisting}

    \item When $\ecentre_\icentre \in \centre$ is an installed centre, the sum of the population of
    the cities in $P_\icentre$ plus 10\% of the sum of the population of the cities in $S_\icentre$
    can not exceed its capacity.
    \begin{equation}
    \sum_{\icity = 1}^{\ncity} P_{\icentre, \icity} \cdot p_\icity +
    0.10 \cdot \sum_{\icity = 1}^{\ncity} S_{\icentre, \icity} \cdot p_\icity \le s_\icentre
    \qquad \forall \icentre: 1 \le \icentre \le \ncentre
    \label{cstr:centres-capacities}
    \end{equation}
    
    \begin{lstlisting}
    forall (t in T) {
        sum (c in C)
	    (
	        primary_centre[t][c]*city_pop[c]
	        +
	        0.10*secondary_centre[t][c]*city_pop[c]
        )
    	<= centre_cap[t];
    }
    \end{lstlisting}

    \item The distance between each pair of installed centres has to be greater or equal than $D$. This
    may be understood as allowing the installation of at most one centre for every pair of locations that
    are strictly closer than $D$.
    \begin{equation}
    \sum_{\icentre = 1}^{\ncentre} ( I_{\icentre, \iloc_1} + I_{\icentre, \iloc_2} )
    \le
    \frac{d_{\iloc_1, \iloc_2}}{D} + 1
    \qquad
    \forall \iloc_1, \iloc_2: 1 \le \iloc_1 < \iloc_2 \le \nloc
    \label{cstr:centres-not-too-close}
    \end{equation}
    To avoid precision issues we will use the following equivalent expression:
    \[
    D^2 \cdot \sum_{\icentre = 1}^{\ncentre} ( I_{\icentre, \iloc_1} + I_{\icentre, \iloc_2} )
    \le
    d_{\iloc_1, \iloc_2}^2 + D^2
    \]
    
    where $d_{\iloc_1, \iloc_2}^2 =
    (u_{\iloc_1, x} - u_{\iloc_2, x})^2 + (u_{\iloc_1, y} - u_{\iloc_2, y})^2$.
    
    \begin{lstlisting}
    forall (l1 in L, l2 in  L : l1 < l2) {
    	(D^2)*(sum (t in T) (location_centre[t][l1] + location_centre[t][l2]))
    	<=
    	((loc_x[l1] - loc_x[l2])^2 + (loc_y[l1] - loc_y[l2])^2) + D^2;
    }
    \end{lstlisting}

    \item If $\ecentre_\icentre \in \centre$ is an installed centre, the distance between the centre
    and the cities in $S_\icentre$ has to be smaller than the working distance of that centre. Likewise,
    this may be understood as not allowing a primary centre $\iloc$ serving a city $\icity$ from a given
    location $\iloc$ if the distance between the location and the city is too large.
    \begin{equation}
    P_{\icentre, \iloc} < \frac{\omega_{\icentre}}{d_{\iloc, \icity}}
    \qquad
    \forall \iloc, \icity, \icentre:
    1 \le \iloc \le \nloc,\;
    1 \le \icity \le \ncity,\;
    1 \le \icentre \le \ncentre
    \label{cstr:distance-centre-city-primary}
    \end{equation}
    
    To avoid precision issues we will be using the following equivalent expression:
    \[
    d_{\iloc, \icity}^2 \cdot P_{\icentre, \iloc} < \omega_{\icentre}^2
    \]
    where
    $d_{\iloc, \icity}^2 = (u_{\iloc, x} - v_{\icity, x})^2 + (u_{\iloc, y} - v_{\icity, y})^2$.

    \begin{lstlisting}
    forall (c in C, l in L) {
    	forall (t in T) {
    	    primary_centre[t][c]
    	    *
    		((city_x[c] - loc_x[l])^2 + (city_y[c] - loc_y[l])^2)
    		<= (work_dist[t])^2;
    	}
    }
    \end{lstlisting}

    \item If $\ecentre_\icentre \in \centre$ is an installed centre, the distance between the centre
    and the cities in $P_\icentre$ has to be smaller than three times the working distance of that
    centre. Likewise, this may be understood as not allowing a secondary centre $\iloc$ serving a
    city $\icity$ from a given location $\iloc$ if the distance between the location and the city
    is too large.
    \begin{equation}
    S_{\icentre, \iloc} < \frac{3\omega_{\icentre}}{d_{\iloc, \icity}}
    \qquad
    \forall \iloc, \icity, \icentre:
    1 \le \iloc \le \nloc,\;
    1 \le \icity \le \ncity,\;
    1 \le \icentre \le \ncentre
    \label{cstr:distance-centre-city-secondary}
    \end{equation}
    
    To avoid precision issues we will be using the following equivalent expression:
    \[
    d_{\iloc, \icity}^2 \cdot S_{\icentre, \iloc} < 9\omega_{\icentre}^2
    \]
    where
    $d_{\iloc, \icity}^2 = (u_{\iloc, x} - v_{\icity, x})^2 + (u_{\iloc, y} - v_{\icity, y})^2$.
    
    \begin{lstlisting}
    forall (c in C, l in L) {
    	forall (t in T) {
    	    secondary_centre[t][c]
    	    *
    		((city_x[c] - loc_x[l])^2 + (city_y[c] - loc_y[l])^2)
    		<= (3*work_dist[t])^2;
    	}
    }
    \end{lstlisting}

\end{enumerate}

Some constraints are used to relate decision variables with other decision variables:

\begin{itemize}
    
    \item[-] Constraints \ref{cstr:one-loc-one-centre-1} and \ref{cstr:one-loc-one-centre-2} ensure the
    one-to-one relationship between centres and locations: a centre can only be installed in at most one
    location and one location can have at most one centre installed.
    
    \item[-] Constraint \ref{cstr:is-centre-installed} relates decision variables 
    \ref{dec-var:centre-installed} and \ref{dec-var:centre-location} so as to be able to define the
    objective function.
    
    \item[-] Constraint \ref{cstr:centres-not-too-close} basically models the following implication:
    \[
    d_{\iloc_1, \iloc_2} < D
    \Longrightarrow
    \sum_{\icentre = 1}^{\ncentre} ( I_{\icentre, \iloc_1} + I_{\icentre, \iloc_2} ) \le 1
    \]
    $\forall \iloc_1, \iloc_2: 1 \le \iloc_1 < \iloc_2 \le \nloc$
    and it does so for the following reason: when it is true that $d < D$ the quotient of the
    distances is smaller than 1 ($d / D < 1$). Therefore, bounding the left hand-side of the
    inequation with $d / D + 1$ allows only two possible integer values: $0$ and $1$ (one or
    none of the two locations can have a centre installed). Similarly, when $d \ge D$ the
    quotient will allow for three different possible values $0$, $1$ and $2$ because
    $d / D \ge 1 \Longrightarrow d / D + 1 \ge 2$ (none, one or the two locations can have a
    centre installed). All these arguments also apply for the cases when $d^2 < D^2$ and
    $d^2 \ge D^2$.
    
    \item[-] Constraints \ref{cstr:distance-centre-city-primary} and 
    \ref{cstr:distance-centre-city-secondary} are obtained applying a similar procedure to obtain
    constraint \ref{cstr:centres-not-too-close}. They are both modelling the following implications:
    \[
    \omega_{\icentre} < d_{\iloc, \icity} \Longrightarrow P_{\icentre, \iloc} = 0,
    \quad
    3 \cdot \omega_{\icentre} < d_{\iloc, \icity} \Longrightarrow S_{\icentre, \iloc} = 0
    \]
    $\forall \iloc, \icity, \icentre: 1 \le \iloc \le \nloc,\; 1 \le \icity \le \ncity,\;
    1 \le \icentre \le \ncentre$. Using this and very similar arguments to the ones used to obtain
    constraint \ref{cstr:centres-not-too-close} we can obtain those that relate primary and secondary
    centres with the cities they serve.
    
\end{itemize}

Some other constraints were added just to avoid silly solutions:

\begin{itemize}
    
    \item[-] Constraint \ref{cstr:inst-centres-must-serve} avoids installing centres with cost
    $i_\icentre = 0$ at empty locations and having these centres serving no city.
    
    \item[-] Constraint \ref{cstr:not-installed-not-prim-sec} will ensure that a city is served only
    by installed centres. This way all cities will be served by centres the cost of which will be used
    in the objective function.
    
\end{itemize}

Constraints \ref{cstr:all-cities-served}, \ref{cstr:centre-not-too-busy}, \ref{cstr:centres-capacities}
are extracted directly from the statement and are self-explanatory.

