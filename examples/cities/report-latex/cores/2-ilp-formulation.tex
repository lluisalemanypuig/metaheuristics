\newcounter{decisionVariables}

\section{Integer Linear Programming formulation}
\label{sec:ILP-form}

In this section is described all the information regarding the ILP formulation of the problem: the
variables, objective function, the constraints to model the problem, and the implementation in OPL.

\subsection{Input data}
\label{sec:ILP-form:inp-data}

The input data that makes up an instance of the problem, extracted from the statement, is referenced in
the OPL code using the following names:

\begin{itemize}
    \item[--] The ranges $1, \cdots, \nloc$, $1, \cdots, \ncity$ and $1, \cdots, \ncentre$, where
    $\nloc$, $\ncity$ and $\ncentre$ are the number of locations, cities and centre types respectively:
    
    \begin{lstlisting}
    range L = 1..nLocations;
    range C = 1..nCities;
    range T = 1..nCentres;
    \end{lstlisting}

    \item[--] Location coordinates $u_\iloc$
    
    \begin{lstlisting}
    float loc_x[l in L];
    float loc_y[l in L];
    \end{lstlisting}
    
    \item[--] City coordinates $v_\icity$:
    
    \begin{lstlisting}
    float city_x[c in C];
    float city_y[c in C];
    \end{lstlisting}
    
    \item[--] City population $p_\icity$:
    
    \begin{lstlisting}
    int city_pop[c in C];
    \end{lstlisting}
    
    \item[--] Centre's working distance $\omega_\icentre$:
    
    \begin{lstlisting}
    float work_dist[t in T];
    \end{lstlisting}
    
    \item[--] Centre's capacity $s_\icentre$:
    
    \begin{lstlisting}
    int centre_cap[t in T];
    \end{lstlisting}
    
    \item[--] Centre's installation cost $i_\icentre$:
    
    \begin{lstlisting}
    float instal_cost[t in T];
    \end{lstlisting}
    
    \item[--] Distances between locations, and between locations and cities.
    
    \begin{lstlisting}
    float dist_ll[l1 in L][l2 in L];
    float dist_lc[l in L][c in C];
    \end{lstlisting}
    
\end{itemize}

The values of each of these variables are read from the input with the only exception of the last two.
These are calculated in a preprocessing block and are meant to make the code more readable. Here is the
preprocessing block:

\begin{lstlisting}
execute {
	for (var l1 = 1; l1 <= nLocations; ++l1) {
		for (var l2 = 1; l2 <= nLocations; ++l2) {
			dist_ll[l1][l2] = Math.sqrt(
			    Math.pow(loc_x[l1] - loc_x[l2], 2) +
			    Math.pow(loc_y[l1] - loc_y[l2], 2)
            );
		}
		
		for (var c = 1; c <= nCities; ++c) {
			dist_lc[l1][c] = Math.sqrt(
			    Math.pow(loc_x[l1] - city_x[c], 2) +
			    Math.pow(loc_y[l1] - city_y[c], 2)
            );
		}
	}
}
\end{lstlisting}

\subsection{Decision variables}
\label{sec:ILP-form:dec-var}

For the ILP formulation of this problem the following decision variables have been defined. The name
each of these decision variables is given and the OPL code is also specified.

\begin{enumerate}
    \item $I_{\iloc, \icentre}$: ``centre of type $\ecentre_\icentre$ is installed in location
    $\eloc_\iloc$" -
    ($\mathbb{B}$, $\forall \iloc, \icentre: 1 \le \iloc \le \nloc, 1 \le \icentre \le \ncentre$)
    \label{dec-var:centre-installed}
    \stepcounter{decisionVariables}
        
    \begin{lstlisting}
    dvar boolean location_centre[l in L][t in T];
    \end{lstlisting}
    
    \item $LP_{\iloc, \icity}$: ``location $\eloc_\iloc$ serves city $\ecity_\icity$ with a primary
    role" - ($\mathbb{B}$, $\forall \iloc, \icity: 1 \le \iloc \le \nloc,\:
    1 \le \icity \le \ncity$)
    \label{dec-var:location-primary-centre}
    \stepcounter{decisionVariables}
    
    \begin{lstlisting}
    dvar boolean location_pcity[l in L][c in C];
    \end{lstlisting}

    \item $LS_{\iloc, \icity}$: ``location $\eloc_\iloc$ serves city $\ecity_\icity$ with a secondary
    role" - ($\mathbb{B}$, $\forall \iloc, \icity: 1 \le \iloc \le \nloc,\:
    1 \le \icity \le \ncity$)
    \label{dec-var:location-secondary-centre}
    \stepcounter{decisionVariables}
    
    \begin{lstlisting}
    dvar boolean location_scity[l in L][c in C];
    \end{lstlisting}
    
\end{enumerate}

\subsection{Objective function}
\label{sec:ILP-form:obj-func}

The value to minimise is the cost of centre installation. Therefore, we will be using the
following objective function:

\begin{equation}
\text{minimise }
\sum_{\icentre = 1}^{\ncentre}
    \left(
        i_\icentre \cdot 
        \sum_{\iloc = 1}^{\nloc} I_{\iloc, \icentre}
    \right)
\label{ILP:objective-function}
\end{equation}

that involves the installed centre types $I_{\iloc, \icentre}$ and the corresponding installation
cost $i_\icentre$.

\hfill

The implementation in OPL is the following:

\begin{lstlisting}
minimize sum (t in T) instal_cost[t]*( sum (l in L) location_centre[l][t] );
\end{lstlisting}

\subsection{Constraints}
\label{sec:ILP-form:constraints}

The problem is solved using the following constraints.

\begin{enumerate}
\setcounter{enumi}{\value{equation}}

    \item A location can have at most one centre installed.
    \begin{equation}
    \sum_{\icentre = 1}^{\ncentre} I_{\iloc, \icentre} \le 1
    \qquad \forall \iloc: 1 \le \iloc \le \nloc
    \label{cstr:one-loc-one-centre-1}
    \end{equation}
    
\begin{lstlisting}
forall (l in L) {
	sum (t in T) location_centre[l][t] <= 1;
}
\end{lstlisting}
    
    \item If a location is serving one or more cities then it must have a centre installed.
    \begin{equation}
    C \cdot \sum_{\icentre = 1}^{\ncentre} I_{\iloc, \icentre}
    \ge
    \sum_{\icity = 1}^{\ncity} (LP_{\iloc, \icity} + LS_{\iloc, \icity})
    \qquad \forall \iloc: 1 \le \iloc \le \nloc
    \label{cstr:serving-location-with-centre}
    \end{equation}
    
\begin{lstlisting}
forall (l in L) {
	nCities*( (sum (t in T) location_centre[l][t]) )
	>= 
	(sum (c in C) (location_pcity[l][c] + location_scity[l][c]));
}
\end{lstlisting}
	
	This constraint is basically modelling the following implication:
	\[
	\sum_{\icity = 1}^{\ncity} LP_{\iloc, \icity} + LS_{\iloc, \icity} \ge 1
	\Longrightarrow
	I_{\iloc, \icentre} = 1
	\]
	
	\item Each city has to be served exactly by two locations: one with a primary role and
	another with a secondary role.
    
    \begin{equation}
    \sum_{\iloc = 1}^{\nloc} LP_{\iloc, \icity} = 1, \quad
	\sum_{\iloc = 1}^{\nloc} LS_{\iloc, \icity} = 1
	\qquad \forall \icity: 1 \le \icity \le \ncity
	\label{cstr:cities-served-by-location}
    \end{equation}
    
\begin{lstlisting}
forall (c in C) {
	sum (l in L) location_pcity[l][c] == 1;
	sum (l in L) location_scity[l][c] == 1;
}
\end{lstlisting}
    
    \item A location can not serve a city as both primary and secondary centre.
    
    \begin{equation}
    LP_{\iloc, \icity} + LS_{\iloc, \icity} \le 1
    \qquad \forall \iloc, \icity:
    1 \le \iloc \le \nloc,\;
    1 \le \icity \le \ncity
    \label{cstr:centre-not-too-busy}
    \end{equation}
    
\begin{lstlisting}
forall (l in L, c in C) {
	location_pcity[l][c] + location_scity[l][c] <= 1;
}
\end{lstlisting}
    
    This is modelling the following implications:
    \[
    LP_{\iloc, \icity} = 1 \iff LS_{\iloc, \icity} = 0, \qquad
    LP_{\iloc, \icity} = 0 \iff LS_{\iloc, \icity} = 1
    \]
    
    \item The sum of the population of those cities a location serves with a primary centre
    plus 10\% of the sum of the population of those cities it serves with a secondary centre
    can not exceed its capacity.
    
    \begin{equation}
    \sum_{\icity = 1}^{\ncity} LP_{\iloc, \icity} \cdot p_{\icity} +
    0.10 \cdot \sum_{\icity = 1}^{\ncity} LS_{\iloc, \icity} \cdot p_{\icity}
    \le
    \sum_{\icentre = 1}^{\ncentre} I_{\iloc, \icentre} \cdot s_{\icentre}
    \qquad \forall \iloc: 1 \le \iloc \le \nloc
    \label{cstr:centre-not-overbooked}
    \end{equation}
    
\begin{lstlisting}
forall (l in L) {
	sum (c in C) location_pcity[l][c]*city_pop[c]
	+
	0.10*( sum (c in C) location_scity[l][c]*city_pop[c] )
	<=
	sum (t in T) location_centre[l][t]*centre_cap[t];
}
\end{lstlisting}
    
    Assume $P_{\iloc}$ and $S_{\iloc}$ to be
    the sets of cities a location $\iloc \in \loc$ serves with a primary and a secondary
    role respectively. Formally, using the previously defined boolean variables:
    
    \[
    P_{\iloc} = \{\icity \in \city \;|\; LP_{\iloc, \icity} = 1 \},
    S_{\iloc} = \{\icity \in \city \;|\; LS_{\iloc, \icity} = 1 \}
    \]
    
    Constraint \ref{cstr:centre-not-overbooked} is modelling the following implication:
    
    \[
    I_{\iloc, \icentre} = 1 \Longrightarrow
    \sum_{\icity \in P_\iloc} p_c + 0.10 \cdot \sum_{\icity \in S_\iloc} p_c
    \le
    \omega_{\icentre}
    \]
    
    $\forall \iloc, \icentre: 1 \le \iloc \le \nloc,\; 1 \le \icentre \le \ncentre$.
    
    \item Centres cannot be installed in locations that are at a distance smaller than D.
    
    \begin{equation}
    \sum_{\icentre = 1}^{\ncentre} I_{\iloc_1, \icentre} + I_{\iloc_2, \icentre} \le 1
    \qquad
    \forall \iloc_1, \iloc_2:
    1 \le \iloc_1 < \iloc_2 \le \nloc,\; : \;
    d_{\iloc_1, \iloc_2} < D
    \end{equation}
    
    where $d_{\iloc_1, \iloc_2}^2 =
    (u_{\iloc_1, x} - u_{\iloc_2, x})^2 + (u_{\iloc_1, y} - u_{\iloc_2, y})^2$ is the
    squared distance between the locations $\iloc_1$ and $\iloc_2$.
    
\begin{lstlisting}
forall (l1 in L, l2 in L : l1 < l2) {
	if (dist_ll[l1][l2] < D) {
		sum (t in T) (location_centre[l1][t] + location_centre[l2][t]) <= 1;
	}
}
\end{lstlisting}
    
    This is modelling the following implication:
    
    \[
    d_{\iloc_1, \iloc_2} < D \Longrightarrow
    (I_{\iloc_1, \icentre} = 0 \wedge I_{\iloc_2, \icentre} = 1) \vee
    (I_{\iloc_1, \icentre} = 1 \wedge I_{\iloc_2, \icentre} = 0)
    \]
    
    $\forall \iloc_1, \iloc_2, \icentre:
    1 \le \iloc_1 < \iloc_2 \le \nloc, \; 1 \le \icentre \le \ncentre$.
    
    \item The distance between the location and the cities it serves with a primary role can not be
    greater than the working distance of that centre.
    
    \begin{equation}
    LP_{\iloc, \icity} \cdot d_{\iloc, \icity} \le
    \sum_{\icentre = 1}^{\ncentre} I_{\iloc, \icentre} \cdot \omega_{\icentre}
    \qquad
    \forall \iloc, \icity:
    1 \le \iloc \le \nloc,\;
    1 \le \icity \le \ncity
    \label{cstr:working-distance-primary}
    \end{equation}
    
    where $d_{\iloc, \icity}^2 =
    (u_{\iloc, x} - v_{\icity, x})^2 + (u_{\iloc, y} - v_{\icity, y})^2$ is the squared distance
    between location $\iloc$ and city $\icity$.
    
\begin{lstlisting}
forall (l in L, c in C) {
	location_pcity[l][c]*dist_lc[l][c]
	<=
	sum (t in T) location_centre[l][t]*work_dist[t];
}
\end{lstlisting}
	
	\hfill
    
    This constraint is modelling the following implication:
    \[
    I_{\iloc, \icentre} = 1 \wedge LP_{\iloc, \icity} = 1
    \Longrightarrow d_{\iloc, \icity} \le \omega_\icentre
    \]
	
	$\forall \iloc, \icity, \icentre: 1 \le \iloc \le \nloc,\; 1 \le \icity \le \ncity,\;
	1 \le \icentre \le \ncentre$.
	
	\item The distance between the centre and the cities it serves with a secondary role can not be
	greater than three times the working distance of that centre.
		
\begin{equation}
    LS_{\iloc, \icity} \cdot d_{\iloc, \icity} \le
    \sum_{\icentre = 1}^{\ncentre} I_{\iloc, \icentre} \cdot 3\omega_{\icentre}
    \qquad
    \forall \iloc, \icity:
    1 \le \iloc \le \nloc,\;
    1 \le \icity \le \ncity
    \label{cstr:working-distance-secondary}
    \end{equation}
    
    where $d_{\iloc, \icity}^2 =
    (u_{\iloc, x} - v_{\icity, x})^2 + (u_{\iloc, y} - v_{\icity, y})^2$ is the distance between
    the location $\iloc$ and the city $\icity$.
    
\begin{lstlisting}
forall (l in L, c in C) {
	location_scity[l][c]*dist_lc[l][c]
	<=
	sum (t in T) location_centre[l][t]*3*work_dist[t];
}
\end{lstlisting}
	
	\hfill
    
    This constraint is modelling the following implication:
    \[
    I_{\iloc, \icentre} = 1 \wedge LS_{\iloc, \icity} = 1
    \Longrightarrow d_{\iloc, \icity} \le 3\omega_\icentre
    \]
	
	$\forall \iloc, \icity, \icentre: 1 \le \iloc \le \nloc,\; 1 \le \icity \le \ncity,\;
	1 \le \icentre \le \ncentre$.
	    
\end{enumerate}



